\chapter{Exemplo de Capítulos e Seções}
\label{chap:exemplo-de-capitulos-e-secoes}

Este capítulo descreve como você deve formatar os títulos do seu trabalho, ensinando também como você deve usá-los. O capítulo deve ser caixa alta, 12pt e com negrito. Para fazer um capítulo, você deve fazer:

\verb!\chapter{Escreva aqui o nome do Capítulo}!

\section{Seção Secundária}
As seções Secundárias deve ser caixa alta, 12pt e sem negrito. Para fazer uma seção Secundária, você deve fazer:

\verb!\section{Escreva aqui o nome da seção}!

\subsection{Seção Terciária}
As seções Terciárias deve ser caixa alta e baixa, 12pt e com negrito. Para fazer uma seção Terciária, você deve fazer:

\verb!\subsection{Escreva aqui o nome da seção}!

\subsubsection{Seção Quaternária}
As seções Quaternárias deve ser caixa alta e baixa, 12pt e sem negrito. Para fazer uma seção Quaternária, você deve fazer:

\verb!\subsubsection{Escreva aqui o nome da seção}!

\subsubsubsection{Seção Quinária}
As seções Quinárias deve ser caixa alta e baixa, 12pt e com itálico. Para fazer uma seção Quinária, você deve fazer:

\verb!\subsubsubsection{Escreva aqui o nome da seção}!