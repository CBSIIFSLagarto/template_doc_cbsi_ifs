% [puro-uff-rfm: 1974] reunião do departamento de física e matemática
% Nomes dos pontos de pauta:
%   projeto de monitoria
%   consulta para escolha de representantes aos conselhos da UFF
%   vista ao projeto do regimento do colegiado de unidade
%   condições de trabalho no Polo
%   confraternização de fim de ano
%   
%   Aprovação do Parecer Conclusivo do Estágio Probatório do Prof. Reginaldo.
%   Aprovação das datas e distribuição de aulas do Cálculo Zero

\documentclass[12pt,a4paper]{ata} 
\usepackage[brazil]{babel}
%\usepackage[latin1]{inputenc}
\usepackage[utf8]{inputenc} 
\usepackage{amssymb}
\usepackage[T1]{fontenc}
\usepackage[a4paper, top=2.3in,bottom=1.2in,right=1in,left=1in,headheight=114.4pt]{geometry}


\usepackage[nogroupskip,nonumberlist,acronym,automake]{glossaries}                % Permite fazer o glossario

\newcommand{\dummyText}{
	The European languages are members of the same family. Their separate existence is a myth. For science, music, sport, etc, Europe uses the same vocabulary. 
}

\newcommand{\imprimirlistadeabreviaturasesiglas}[1][0]{ %
	\printglossary[type=acronym,style=long,nonumberlist,title={Lista de Abreviaturas e Siglas\vspace{-0.3cm}}] %
	\ifthenelse{1 = #1}{\pagebreak}{}
}

% Organiza e gera a lista de abreviaturas, simbolos e glossario
\makeglossaries
% custom footers and headers
\usepackage{fancyhdr}
\pagestyle{fancy}

\renewcommand{\footrulewidth}{1pt}

\lhead{}
\chead{
	{\includegraphics[width=.1\textwidth]{brasao}} \\
	{Ministério da Educação}\\
	{Secretaria de Educação Profissional e Tecnológica}\\
	{Instituto Federal de Educação, Ciência e Tecnologia de Sergipe} \\
	{\deptoMM \space -- \deptoM \space -- Campus Lagarto}	
}
\rhead{}
\lfoot{}
\cfoot{
	{\scriptsize \sffamily Rua Cauby, nº 523, Bairro Jardim Campo Novo}\\
	{\scriptsize \sffamily CEP: 49400-000 - Telefone Geral (79) 3321-1500 - Lagarto - SE - Brasil}
}
\rfoot{
	{\scriptsize \sffamily \vspace{2pt} Página \thepage / \pageref{LastPage}}
 }

\newcommand{\depto}[2]{
	\newcommand{\deptoMM}{#1}
	\newcommand{\deptoM}{#2}
}

\newcommand{\cabecalho}[1]{
	%	\noindent
	\hfill	
	{\bf \textbf{Assunto}:}  #1		
	\vspace{.2cm}
	
	\title{#1}
	
	% \vskip -5ex
	
	\linenumbers
}







\newacronym{CBSI}{CBSI}{Coordenadoria de Sistemas de Informação} % exemplo de sigla
\newacronym{IFS}{IFS}{Instituto Federal de Sergipe} % exemplo de sigla
\newacronym{IEEE}{IEEE}{Instituto Federal de Sergipe} % exemplo de sigla
\newcommand{\nomePessoa}[1]{\glsdesc{#1} - matrícula \glssymbol{#1}}

\newacronym[symbol={147}]{p.Pessoa1}{Nome da Pessoa 1}{Nome Longo da Pessoa 1} 
\newacronym[symbol={258}]{p.Pessoa2}{Nome da Pessoa 2}{Nome completo da pessoa 2}
\newacronym[symbol={369}]{p.Pessoa3}{Pessoa 3}{Nome completo da pessoa 3}
\newacronym[symbol={123}]{p.Pessoa4}{Pessoa 4}{Nome completo da pessoa 4}


\depto{Coordenadoria de Sistemas de Informação}{CBSI}

\usepackage{ifmtarg}

\begin{document}

\def\intriangles#1{$\blacktriangleleft$#1$\blacktriangleright$}
\def\dontshowTs{\def\TE##1{}}
\def    \showTs{\def\TE##1{\intriangles{##1}}}
   
\showTs % Mostrar o horário em que foi dito
\dontshowTs % Ocultar horário 

\label{Data da reunião}
\dia{12}
\mes{07}
\ano{2018}

\label{Tipo da reunião}
\deftiporeuniao{Ordinária}
% Advérbio correspondente ao tipo da reunião.
\deftiporeuniaoadv{ordinariamente}


\label{Presidente da banca}
% Quem presidiu a reunião (sem título).
\defpresidiu{\gls{p.Pessoa3}}

\label{Cargo do presidente da banca}
% Cargo de quem presidiu a reunião.
\defcargopresidiu{coordenador do curso}

\label{Secretario}
% Quem secretariou (sem título).
\defsecretariou{\gls{p.Pessoa1}}

\label{Lista Prof. Presentes}
% Lista de presentes. NÃO terminar com ponto.
\defpresentes{
	\nomePessoa{p.Pessoa1}, 
	\nomePessoa{p.Pessoa2}} %O último da lista deve ser acompanhado com o fecha }

\label{Lista Prof. Justificados}
% Lista de ausentes justificados. NÃO terminar com ponto.
\defjustif{
	\nomePessoa{p.Pessoa3},
	{Pessoa convidada 01}}

\label{Lista Prof. Ausentes}
% Lista de ausentes. NÃO terminar com ponto.
\defausentes{
	\nomePessoa{p.Pessoa4},
	{Pessoa convidada ausente}}

\label{Cabecalho}
%% Cabeçalho com quebras de linhas manuais. 
%% Basicamente sempre o mesmo.
%% O comando \hoje gera a data (em numerais). 
\cabecalho{%
	Ata da Reunião \tiporeuniao{} da  
	Coordenação de Sistemas de Informação,
	do Instituto de Federal de Sergipe,
	Campus Lagarto
	realizada em \hoje.%
}
\label{Dica: Como adicionar um ponto}
% Detalhamento da pauta. Cada ponto é um ambiente da forma
%
%    \begin{ponto}{Nome do ponto}
%    
%      Corpo do ponto.
%    
%    \end{ponto}
%
% O nome do ponto não deve terminar com pontuação. O corpo sim.
% Quebras de linha e listas de itens podem ser inseridas à vontade,
% mas a ata inteira será diagramada como um único parágrafo.
\label{Dica: Adicionar o horário em que foi dito algo}
% Para adicionar o tempo em que foi dito, usar:
%
%  \showTs % ativa a exibição do horário
%  \begin{ponto}{Item 1 da ata de reunião
%  \T{10:50} Falou que....
%  \end{ponto}
%  \dontshowTs
%
%  Isso permite que seja incluída o horário em que a fala foi realizada, sanbendo que
%
%  \dontshowTs - Desativa a exibição do horário
%  \showTs - Ativa a exibição do horário

\begin{ponto}{Indicação de uma pessoa para secretariar a reunião}
  O~\prof \secretariou \space se ofereceu para secretariar a reunião.
\end{ponto}

\begin{ponto}{Aprovação da ata da reunião anterior}
\TE{2:00}  A ata da reunião anterior foi aprovada por unanimidade.
\end{ponto}

\begin{ponto}{Aprovação do Parecer Conclusivo de Estágio Probatório do
  \prof \gls{p.Pessoa1}
}

\TE{2:13} A \profa \gls{p.Pessoa2} circula entre os presentes
todos os pareceres de relatórios de estágio probatório do...

\end{ponto}



\begin{ponto}{Projetos de monitoria 2013}

\dummyText

\end{ponto}

\begin{ponto}{Condições de trabalho no Polo}
	\dummyText
\end{ponto}

\begin{ponto}{Solicitação de espaço físico para alunos}
	\dummyText
\end{ponto}



\begin{ponto}{Aprovação das datas e distribuição de aulas do Cálculo Zero}
	\dummyText
\end{ponto}


% antigos
\begin{ponto}{Informes}

  \begin{enumerate}

  \item Falta d'água: \dummyText.

  \item Aparelhos de ar-condicionado: \dummyText.

  \item Papel~A$4$: \dummyText.

  \item Marimbondos: \dummyText.

  \item Princípio de incêndio em transformador: \dummyText.

  \item Falta de informações: A~\profa \gls{p.Pessoa3} \dummyText.

  \item Falecimento de aluno da Ciência da Computação: que isso não aconteça.

  \item \dummyText. 

  \item Estágio probatório do~\prof: o~\prof \gls{p.Pessoa2} informou que o~processo de aprovação do~estágio probatório do~\prof já foi homologado.

  \item Apresentação de \dummyText.

  \item Gratificação para chefia: \dummyText.

  \end{enumerate}

\end{ponto}

\label{Aprovação em ata}
%\begin{ponto}{Aprovação de ata}
%  A~plenária aprovou a~ata da~reunião ordinária de~$11$ de~outubro
%  de~$2012$ por~unanimidade.
%\end{ponto}


%   ____                             _               _        
%  / ___|___  _ __ _ __   ___     __| | __ _    __ _| |_ __ _ 
% | |   / _ \| '__| '_ \ / _ \   / _` |/ _` |  / _` | __/ _` |
% | |__| (_) | |  | |_) | (_) | | (_| | (_| | | (_| | || (_| |
%  \____\___/|_|  | .__/ \___/   \__,_|\__,_|  \__,_|\__\__,_|
%                 |_|                                         
%%%%%%%%%%%%%%%%%%%%%%%%%%%%%%%%%%%%%%%%%%%%%%%%%%%%%%%%%%%%%%%%
%
% Corpo da ata. Basicamente sempre o mesmo. Verificar se a pauta foi
% aprovada por unanimidade. Se não houver faltantes, eliminar a(s)
% linha(s) correspondente(s).
\label{Introdução da ata}
\hojeporextenso, reuniram-se \tiporeuniaoadv{} os professores da Coordenadoria de Sistemas de Informação (CBSI) do Instituto de Ciência e Tecnologia de Sergipe.
%
\presentes
%
%Compareceram os professores \presentes. 
%
\justificados
%
\ausentes
%
Abrindo a reunião, o \prof \presidiu, \cargopresidiu, apresentou a
seguinte proposta de pauta, que foi aprovada por unanimidade pela
plenária:
%
\pauta
%
A reunião se desenvolveu conforme se segue:
% 
\desenvolvimento
% 
Não havendo mais a tratar, a reunião foi encerrada, da qual eu,
\secretariou, lavrei a presente ata, que vai assinada por mim, pelo 
\cargopresidiu \space (\presidiu) e pelos demais participantes.
%

\imprimirRegiaoAssinaturaPresentes

\end{document}


%\documentclass{article}
%\begin{document}
%	\imprimirRegiaoAssinaturaPresentes
%\end{document}

%
%\assinatura{\glsdesc{p.Pessoa1}}
%\assinatura{Lagarto}
%
%\assinatura{Lagarto}
%\assinatura{Ivar Nesje}
%\assinatura{Team mate's name}


