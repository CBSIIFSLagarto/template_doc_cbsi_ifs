\chapter{Conclusão}
\label{chap:conclusao}

\lipsum[2]

Ola \cite{lamport1986latex}
\cite{Maia2011}

\begin{table}[htb]
	\IBGEtab{%
		\caption{Um Exemplo de tabela alinhada que pode ser longa ou curta,
			conforme padrão IBGE.}%
		\label{tabela-ibge}
	}{%
	\begin{tabular}{ccc}
		\toprule
		Nome & Nascimento & Documento \\
		\midrule \midrule
		Maria da Silva & 11/11/1111 & 111.111.111-11 \\
			Maria da Silva & 11/11/1111 & 111.111.111-11 \\
				Maria da Silva & 11/11/1111 & 111.111.111-11 \\
		\bottomrule
	\end{tabular}%
}{%
\fonte{Produzido pelos autores}%
\nota{Esta éuma nota, que diz que os dados são baseados na
	regressão linear.}%
\nota[Anotações]{Uma anotação adicional, seguida de várias outras.}%
}
\end{table}

\cite{Huetal2000}

\section{Exemplo de Algoritmos}
\lipsum[2]

\vspace{\onelineskip}
\begin{algorithm}[H]
	\Entrada{o proprio texto}
	\Saida{como escrever algoritmos com \LaTeX2e }
	\Inicio{
		inicializa\c{c}\~ao\;
		\Repita{fim do texto}{
			leia o atual\;
			\Se{entendeu}{
				vá para o próximo\;
				próximo se torna o atual\;}
			\Senao{volte ao início da seção\;}
		}
	}
	\caption{Como escrever algoritmos no \LaTeX2e}
\end{algorithm}
\vspace*{\onelineskip}

\lipsum[2]
%\begin{algorithm}[H]
%	\Entrada{o proprio texto}
%	\Saida{como escrever algoritmos com \LaTeX2e }
%	\Inicio{
%		inicializa\c{c}\~ao\;
%		\Repita{fim do texto}{
%			leia o atual\;
%			\Se{entendeu}{
%				vá para o próximo\;
%				próximo se torna o atual\;}
%			\Senao{volte ao início da seção\;}
%		}
%	}
%	\caption{Exemplo de Algoritmo Versao 02}
%\end{algorithm}

%\begin{algorithm}
%	\begin{algorithmic}
%	\Entrada{o proprio texto}
%	\Saida{como escrever algoritmos com \LaTeX2e }	
%	\end{algorithmic}
%\end{algorithm}

\begin{table}[h!]	
	\caption{Internal exon scores}
	\label{tab:internal}
	\centering
	\begin{tabular}{|c|l|l|}
		\hline
		Ranking & Exon Coverage & Splice Site Support\\
		\hline
		E1 & Complete coverage by a single transcript & Both splice sites\\
		E2 & Complete coverage by more than a single transcript & Both splice sites\\
		E3 & Partial coverage & Both splice sites\\
		E4 & Partial coverage & One splice site\\
		E5 & Complete or partial coverage & No splice sites\\
		E6 & No coverage & No splice sites\\
		\hline
	\end{tabular}
%	\legend{\small{Fonte: os autores}}

	%\fontedatabela{14.6cm}{os autores}
\end{table}
\lipsum[2]