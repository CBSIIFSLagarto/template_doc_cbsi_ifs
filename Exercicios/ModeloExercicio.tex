% Fundamentos Matemáticos para Computação 1 - 2015.2
% Avaliação 2: Teoria dos Números, Congruência
% Autor: Patrick Terrematte

\documentclass[11pt,a4paper]{article}
\usepackage{../libs/formatacaoCSharp} % usando formatação de código C#
\usepackage[utf8]{inputenc}     % Comentar para compilar no windows
\usepackage{graphicx}
\usepackage{color}
\usepackage[document]{ragged2e}
\usepackage{outlines}
\renewcommand{\theenumi}{\arabic{enumi}. }
\renewcommand{\labelenumi}{\theenumi}

\renewcommand{\theenumii}{\alph{enumii}. }
\renewcommand{\labelenumii}{\theenumii}

\renewcommand{\theenumiii}{\roman{enumiii}. }
\renewcommand{\labelenumiii}{\theenumiii}

\renewcommand{\theenumiv}{\arabic{enumiv}) }
\renewcommand{\labelenumiv}{\theenumiv}

\usepackage{xspace}
\usepackage[brazil]{babel}
%\usepackage[latin1]{inputenc}
\usepackage{amsfonts, amssymb,amsmath}
\usepackage{array,booktabs}
\usepackage{proof}
\usepackage{fancyhdr}
 \usepackage{paralist}
%\usepackage[pdftex=true,pagebackref=true,bookmarks=true,bookmarksnumbered=true,pdffitwindow=true]{hyperref}
\usepackage{hyperref}
\usepackage{marginnote}
%\usepackage[top=Bcm, bottom=Hcm, outer=Ccm, inner=Acm, heightrounded, marginparwidth=Ecm, marginparsep=Dcm]{geometry}
\usepackage{comment}


\renewcommand{\b}{\textbf}
\newcommand{\noi}{\noindent}
\newcommand{\Z}{\mathbb{Z}}
\newcommand{\N}{\mathbb{N}}

\setlength{\textwidth}{170mm}%
\setlength{\textheight}{279mm}%
\setlength{\topmargin}{-30mm}%
%\setlength{\bottommargin}{ \,}%
\setlength{\oddsidemargin}{-10mm}
\setlength{\evensidemargin}{-20mm}%


\begin{document}
	

	
\frenchspacing
\begin{center}
    \begin{minipage}{1.7cm}
		\begin{center}
			\includegraphics[height=2.0cm]{../libs/logo-ifs-micro.png}
		\end{center}
	\end{minipage}
	\begin{minipage}{11.4cm}
		\begin{center}
				{\small \textsc{Instituto Federal de Sergipe}			\\
						  \textsc{Bacharelado em Sistemas de Informação} \\
                         \textbf{Disciplina:} Estrutura de Dados I\hspace{.65cm}\textbf{Semestre:} 2018.2\\
                          \textsc{Professor: Francisco Rodrigues Santos}\\
                }
		\end{center}
	\end{minipage}
	\begin{minipage}{1.6cm}
		\begin{center}
			\includegraphics[height=2cm]{../libs/CBSI-logo.jpg}
		\end{center}
	\end{minipage}
\end{center}


{\sf
  \begin{center}
    \Large \textbf{--- Lista Encadeada Circular ---}%
  \end{center}
}\bigskip

\setlength{\marginparwidth}{5cm}
\small \noindent \textbf{Nome:}\hspace{0.3cm}\hrulefill \hrulefill

\reversemarginpar
\thispagestyle{empty}\bigskip

%\noindent - {\bf Todas as resoluções devem incluir os cálculos e raciocínios usados para obter a solução.}

%\noindent - Há um ponto extra distribuido na avaliação.

%\noindent - Não é necessário grampear as avaliações.

%\noindent - Escreva as respostas em apenas um lado de cada folha.
%\noindent -  {\bf Mantenha as respostas em sequência.}

%\noindent -  {\bf No cabeçalho da folha de rascunho, \underline{escreva seu nome}.}

%\noindent -  {\bf No rodapé direito, confira a ordem das folhas, e escreva \underline{``Página $n$ de $m$''}, para $n$ e $m \in \N$.}

%\noindent -  {\bf Se for o caso, consideraremos valores já calculados em outras questões, basta indicar e justificar.}


%\noindent -  {\bf Os códigos deverão ser enviados para o \emph{bitbucket} }

\noindent -  {\bf É permitido o uso de códigos próprios.}

\noindent -  {\bf Não será permitido o uso de coleções implementadas pelo Java, como ArrayList.}


%\lstinputlisting[language=java, caption=TAD Lista - considere a complexidade de todos estes métodos sendo igual a 1]{../codigos2017/Listas/src/main/java/TADLista.java}


\begin{outline}[enumerate]
	\1 Elabore um programa utilizando lista encadeada circular para gerenciar a lista de pessoas que estão cadastradas para jogar no arcade.
	
	Sabe-se ainda que: 
	\2 É possível adicionar ou remover elementos em uma lista que já está em uso;
	\2 É mantido o registro de quantas vezes uma pessoa foi chamada para jogar;
	\2 Quando o próximo jogador é chamado, o atual vai para o final da lista;

	\1 O sistema deverá oferecer a opção no menu para:
	\2 obter o próximo jogador através do uso do método \codef{getProximoJogador(): Pessoa} da \codef{lista};
	\2 Remover um jogador (através do nome);
	\2 Adicionar um jogador;
	\2 Pesquisar quantas pessoas estão na frente de um determinado jogador;
	\2 Listar os próximos 3 jogadores;	
\end{outline}

\begin{flushright}	
%Boa prova!
\end{flushright}
\end{document}
